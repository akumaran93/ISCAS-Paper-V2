\documentclass[conference]{IEEEtran}

\usepackage{cite}


\ifCLASSINFOpdf
  \usepackage[pdftex]{graphicx}

  \graphicspath{{../pdf/}{../jpeg/}}

  \DeclareGraphicsExtensions{.pdf,.jpeg,.png}
\else

\fi

\usepackage{caption}
\usepackage{subcaption}

\usepackage{amsmath}
\usepackage{multirow}
\usepackage{multicol}

\usepackage{algorithmic}

\usepackage{array}
\usepackage{color}

\hyphenation{op-tical net-works semi-conduc-tor}


\begin{document}

\title{On-Chip Output Stage Design for a  Continuous Class F Power Amplifier}

\author{\IEEEauthorblockN{Anil Kumar Kumaran\IEEEauthorrefmark{1},
M. D'Avino\IEEEauthorrefmark{2},
Leo C.N. de Vreede\IEEEauthorrefmark{1}, and 
Morteza S. Alavi 
\IEEEauthorrefmark{1}} 
\IEEEauthorblockA{\IEEEauthorrefmark{1} Electronic Circuits and Architecture (ELCA) Research Group, Delft University of Technology \\
	\IEEEauthorblockA{\IEEEauthorrefmark{2} Catena B.V., Delft, The Netherlands} Email: a.k.kumaran@tudelft.nl}}

\maketitle

\begin{abstract}
Continuous Class F (CCF) power amplifiers (PAs) overcome Class F PA's disadvantage of narrow bandwidth by eliminating short-circuit requirements at the $2^{nd}$ harmonic. At the same time, CCF maintains a peak efficiency of \textit{90.7\%}, which is higher than the traditional Class A and B peak efficiency. This paper explains the equations governing the CCF \color{green} output stage and \color{black}  illustrates the step by step procedure to design four different \color{green} CCF \color{black}  output networks using lossless lumped components for a \textit{2.1--2.7 GHz} band.
The design comprising a second harmonic trap and no RF choke is chosen \color{blue} due to its \color{black} flatter real part impedance \color{blue} and \color{black} smaller reactive part at the \color{blue} fundamental, \color{black} and  the least number of on-chip lumped components.
\end{abstract}

\vspace{1mm}
% keywords
\begin{IEEEkeywords}
Continuous class F (CCF), Output matching network, Power amplifier (PA), Harmonic termination, Differential\color{green}-\color{black}mode analysis, Common\color{green}-\color{black} mode analysis. 
\end{IEEEkeywords}

%\vspace{-0.1in}

\ifCLASSOPTIONpeerreview
\begin{center} \bfseries EDICS Category: 3-BBND \end{center}
\fi

\IEEEpeerreviewmaketitle

\section{Introduction}
Today, there is an increased demand for high-speed and low-cost transmissions in 4/5G and wireless connectivity networks. A CMOS-based transmitter (TX) is a viable architecture to address these requirements as it facilitates the implementation of System-on-Chip (SoC) solution at a low cost. However, the TX power amplifier (PA) is the most challenging blocks in making CMOS SoC, because it should offer high-efficiency \color{green} and high radiated power \color{black} while meeting the stringent spectral mask  and in-band linearity requirements of wireless systems even at backed-off power levels. Currently, most of TX architectures employ linear class A/B PAs. Nevertheless, their ideal  peak efficiencies are only \textit{50}/\textit{78\%}, respectively, due to a \color{green} relatively \color{black} large overlap between the drain voltage/current  waveforms (Fig.\ref{fig:wave_VI}a/b).

On the other hand, class F PAs ideally achieve peak efficiency of \textit{100\%} by utilizing harmonic-frequency resonators \color{blue} to short-circuit \color{black} at even harmonics and open-circuit at odd harmonics \color{blue} leading to non-overlap drain voltage/current waveforms (Fig. \ref{fig:ICF_wave_VI}). Nevertheless, in reality,  \color{black} controlling all \color{blue} harmonics is challenging and increases the network complexity and component losses, degrading its passive efficiency. \color{black} Therefore, practical implementation of class F \color{blue} employs \color{black} up to $3^{rd}$ harmonic \color{blue} impedance terminations and \color{black}  achieves \color{blue} a \color{black} peak efficiency of \textit{90.7\%} due to \color{blue} increasing the overlap region between the drain voltage and current \color{black} compared to \color{blue} the \color{black} ideal class F (Fig. \ref{fig:ICF_wave_VI}) \cite{Raab_max_eff}.
\begin{figure}[!t]
\centering
\captionsetup{font=footnotesize}
\begin{subfigure}{0.24\textwidth}
\includegraphics[width=0.9\textwidth]{Images/Intro/ClassA_shaded.jpg}
\caption{Class A}
\label{fig:CA_wave_VI}
\end{subfigure}
\begin{subfigure}{0.24\textwidth}
\includegraphics[width=0.9\textwidth]{Images/Intro/ClassB_shaded.jpg}
\caption{Class B}
\label{fig:CB_wave_VI}
\end{subfigure}
\begin{subfigure}{0.24\textwidth}
\includegraphics[width=0.9\textwidth]{Images/Intro/ClassF.jpg}
\caption{Ideal Class F}
\label{fig:ICF_wave_VI}
\end{subfigure}
\begin{subfigure}{0.24\textwidth}
\includegraphics[width=0.9\textwidth]{Images/Intro/CF_wave_VI_shaded.jpg}
\caption{Practical Class F}
\label{fig:CF_wave_VI}
\end{subfigure}
\caption{The class A/B/F drain  voltage (V)/current (I) waveforms. \color{black}}
\label{fig:wave_VI}
\vspace{-0.25in}
\end{figure}
Generalized drain source\color{blue}-\color{black}voltage ($V_{DS}$) containing all frequencies up to $3^{rd}$ harmonic \cite{Gen_Vds_eqn} is given by:
\begin{equation}
V_{DS}=\underbrace{1}_{\text{DC}}-\underbrace{\frac{2}{\sqrt{3}} \cos \theta}_{\text{fundamental}}+\underbrace{\frac{1}{3 \sqrt{3}} \cos 3 \theta}_{\text{$3^{rd}$ harmonic}}
\label{eqn_CF_V}
\end{equation}
\color{blue} The drain \color{black}  current ($I_{D}$)\color{blue}, which is a half-sine wave, \color{black} is given by
\begin{equation}
I_{D}=\underbrace{\frac{1}{\pi}}_{\text{DC}}+\underbrace{\frac{1}{2} \cos \theta}_{\text{fundamental}}+\underbrace{\frac{2}{3 \pi} \cos 2 \theta}_{\text{$2^{nd}$ harmonic}}-\underbrace{\frac{2}{15 \pi} \cos 4 \theta}_{\text{$4^{th}$ harmonic}}
\label{eqn_CCF_I}
\end{equation}
\color{blue} The load impedances at the fundamental, $2^{nd}$ and $3^{rd}$ harmonic \color{black}  are represented by $Z_{1f}$, $Z_{2f}$\color{blue} , and $Z_{3f}$, \color{black} respectively.
\begin{equation}
\begin{aligned}
Z_{1f}=\frac{4}{\sqrt{3}}, \hspace{3mm}
Z_{2f}=0, \hspace{3mm}
Z_{3f}=\infty
\end{aligned}
\label{eqn_CF}
\end{equation}
\color{blue} As depicted in Fig. \ref{fig:CF_wave_VI},  one of the advantages of class F PAs is that its normalized peak drain voltage is two. Nonetheless, they suffer from limited operational bandwidth \color{black}  (typically \textit{10\%}) owing to the need for short and open circuit harmonic terminations. Thus, to realize wider bandwidth, the \color{blue} continuous class F (CCF) \color{black} PA has been proposed \cite{CCF_reason}.

\color{blue} In this paper, we propose an on-chip CCF output stage network. This paper is organized as follows. Section \ref{section:CCF} briefly explains the equations governing the operation of CCF. Section \ref{section:ON} discusses the design procedure of different CCF output networks. Section \ref{section:Results} compares the proposed four output stages, and Section \ref{section:Conclusion} concludes this paper. \color{black}

\section{Continuous Class F}
\label{section:CCF}
\vspace{-0.05in}
Compared to class F, the CCF has an imaginary part added at the fundamental and $2^{nd}$ harmonic of \color{blue} the \color{black} voltage waveform. \color{blue} Thus, \color{black} the generalized $V_{DS}$ for CCF is given by \cite{ECCF_Carrubba}:
\begin{equation}
V_{DS}=\underbrace{1}_{\text{DC}}-\underbrace{\frac{2}{\sqrt{3}} \cos \theta-\gamma \sin \theta}_{\text{fundamental}}+\underbrace{\frac{7 \gamma}{6 \sqrt{3}} \sin 2 \theta}_{\text{$2^{nd}$ harmonic}}+\underbrace{\frac{1}{3 \sqrt{3}} \cos 3 \theta}_{\text{$3^{rd}$ harmonic}}
\label{eqn_CCF_V}
\end{equation}
\color{blue} As in class F, \color{black} $I_{D}$ is a half sinusoid \color{blue} (given in (\ref{eqn_CCF_I})). These equations indicate that $V_{DS}$/$I_{D}$ \color{black}  are chosen in such a way that no power is dissipated at higher harmonics. Also, \color{blue} mathematically, \color{black} the $\gamma$ in \color{blue}(\ref{eqn_CCF_V}) does not \color{black}  affect drain efficiency ($\eta_D$). But \color{blue} it affects the peak drain voltage, as depicted in Fig. \ref{fig:CCF_wave_VI}. In reality, however, $I_{D}$ depends on $V_{DS}$, degrading $\eta_D$ as $\gamma$ increases.\color{black} 

\begin{figure}[!t]
\centering
\captionsetup{font=footnotesize}
\includegraphics[width=3.4in, height=2in]{Images/CCF/CCF_wave_VI.jpg}
\caption{$V_{DS}$ and $I_D$ waveform for CCF with \textit{-1.5} $<$ $\gamma$ $<$ \textit{1.5}.}
\label{fig:CCF_wave_VI}
\vspace{-0.25in}
\end{figure}

Fig. \ref{fig:CCF_wave_VI} \color{blue} implies \color{black} that at $\gamma$ = \textit{0}, waveform is like class F. For the $\gamma$ values between \textit{-1} and \textit{1}, $V_{DS}$ remains positive\color{blue}, enabling CCF PA to meet linearity requirements. However, CCF PAs suffer from large peak drain voltage which can be as high as \textit{3.12} times the supply ($V_{DD}$) when $\gamma$ = \textit{-1} or \textit{1}. Using (\ref{eqn_CCF_V}) and (\ref{eqn_CCF_I}), the load impedances of CCF  are calculated \cite{CCFDesign_ali}.\color{black} 
\begin{equation}
\begin{aligned}
Z_{1f}=\frac{4}{\sqrt{3}}+j 2 \gamma, \hspace{2mm}
Z_{2f}=0-j \frac{\pi}{2} \frac{7 \sqrt{3}}{6} \gamma,\hspace{2mm}
Z_{3f}=\infty
\label{eqn_CCF_imp}
\end{aligned}
\end{equation}

In CCF, $Z_{3f}$ remains open-circuited similar to class F. Meanwhile, $Z_{1f}$ and $Z_{2f}$ \color{blue} have \color{black} a reactive part, unlike class F. From  \color{blue} (\ref{eqn_CCF_imp}) \color{black} and Fig. \ref{fig:CCF_SC}, it is observed that if the reactive part of $Z_{1f}$ changes from inductive to capacitive, then the reactive part of $Z_{2f}$ \color{blue}  needs \color{black} to change from capacitive to inductive or vice-versa across the bandwidth to achieve CCF operation. In the next section, the design procedure for the \color{blue}  proposed \color{black}  four output networks is \color{blue} meticulously explained. \color{black} 

\begin{figure}[!t]
\centering
\captionsetup{font=footnotesize}
\includegraphics[width=0.63\linewidth]{Images/CCF/CCF_SC.jpg}
\caption{Variation of $Z_{1f}$, $Z_{2f}$, $Z_{3f}$ for \textit{-1} $<$ $\gamma$ $<$ \textit{1}.}
\label{fig:CCF_SC}
\vspace{-0.05in}
\end{figure}
 

\section{Design of Output Network for CCF}
\label{section:ON}
In this paper, the \color{blue} push-pull (differential) \color{black} structure is chosen for the PA mainly because it \color{blue} decouples the odd harmonics ($1^{st}$/$3^{rd}$) from even harmonic ($2^{nd}$) impedance. \color{black} Moreover, it \color{blue} suppresses supply/substrate noise \color{black} and second-order nonlinearities as well as \color{blue} doubles the RF \color{black} output power. 
\color{blue} In this work, the targeted peak RF power is \textit{27 dBm}  while its operational bandwidth is \textit{2.1 -- 2.7 GHz} with $V_{DD}$ = \textit{2.7 V}. To \color{black} achieve this, a \color{blue} fundamental \color{black} differential impedance of \textit{38.7} $\Omega$ should be presented to the drains of the transistors across the  \color{blue}  \textit{2.1 -- 2.7 GHz} band, which is obtained as follows:\color{black}
\vspace{-0.05in}
\begin{equation}
\begin{aligned}
&V_{FUND}(de)=2*V_{FUND}(se)=2*\frac{2}{\sqrt{3}} V_{DD}=6.24 \hspace{1mm}V\\
&R_{D}(de)=\frac{V_{FUND}(de)^{2}}{2*\text{Peak} \hspace{1mm} P_{OUT}}=38.7 \hspace{1mm} \Omega
\label{eqn_diff_imp}
\end{aligned}
\end{equation}
The requirements of the output network for CCF is exhibited in Tab. \ref{tab:Output_Network_Requirements}. The PA operates in class F mode at the center frequency $\omega_0$ (\textit{2.4 GHz}) with a short at $2\omega_0$ (\textit{4.8 GHz}) and an open at $3\omega_0$ (\textit{7.2 GHz}). But, for all other frequencies, the PA performs in the CCF mode. 

\setlength{\arrayrulewidth}{0.5mm}
\setlength{\tabcolsep}{2pt}
\renewcommand{\arraystretch}{1.5}
\begin{table}[!t]
\centering
\captionsetup{font=footnotesize}
\resizebox{\linewidth}{!}{%
\begin{tabular}{|c|c|c|c|}
\hline
\begin{tabular}[c]{@{}c@{}}\textbf{Class of} \\\textbf{Operation}\end{tabular} & \begin{tabular}[c]{@{}c@{}}\textbf{First Harmonic} \\($\omega$)\end{tabular} & \begin{tabular}[c]{@{}c@{}}\textbf{Second Harmonic} \\(2$\omega$)\end{tabular} & \begin{tabular}[c]{@{}c@{}}\textbf{Third Harmonic} \\(3$\omega$)\end{tabular} \\ \hline
\multirow{2}{*}{\textbf{\begin{tabular}[c]{@{}c@{}}Class F \\ (2.4 GHz)\end{tabular}}} & $\Re(Z_D)$ = 38.7 $\Omega$ & $\Re(Z_D)$= 0 $\Omega$ & \multirow{2}{*}{\begin{tabular}[c]{@{}c@{}}$|Z_D|\approx$ 1000 $\Omega$\end{tabular}} \\ \cline{2-3}
 & $\Im(Z_D)$ = 0 $\Omega$ & $\Im(Z_D)$ = 0 $\Omega$ &  \\ \hline
\multirow{2}{*}{\textbf{\begin{tabular}[c]{@{}c@{}}\\CCF \\ (2.1 - 2.7 GHz)\end{tabular}}} & $\Re(Z_D)$ = 38.7 $\Omega$ & $\Re(Z_D)$ = 0 $\Omega$ & \multirow{2}{*}{\begin{tabular}[c]{@{}c@{}}\\$|Z_D|\approx$ 1000 $\Omega$ \end{tabular}} \\ \cline{2-3}
 & \begin{tabular}[c]{@{}c@{}}$\Im(Z_D)$ $\rightarrow$ + to -\\ OR\\ $\Im(Z_D)$ $\rightarrow$ - to +\end{tabular} & \begin{tabular}[c]{@{}c@{}} $\Im(Z_D)$ $\rightarrow$ - to +\\ OR\\ $\Im(Z_D)$ $\rightarrow$ + to -\end{tabular} &  \\ \hline
\end{tabular}%
 }
\caption{Output network specifications.}
\label{tab:Output_Network_Requirements}
\vspace{-0.25in}
\end{table}

\color{blue} The proposed four different  output networks, which are designed using lossless lumped components, are illustrated in Fig. \ref{fig:Design_A_FC} and \ref{fig:Design_B_C_D}a/b/c. \color{black} 
All the designs \color{blue}  consist of \color{black} a balun and a load capacitance ($C_L$)\color{blue}. The balun converts the balanced (differential) signal to its single-ended companion, whereas $C_L$ adjusts $3^{rd}$ harmonic impedance. The power transistor's drain-source \color{black}  capacitance (assume $C_{DS}=1.87\hspace{1mm}pF$) is absorbed into the output network to reduce its impact on the PA performance. The balun is modeled using \color{blue} an \color{black} ideal transformer, \color{blue} comprising of \color{black} magnetizing inductance ($L_m$), leakage inductance ($L_k$), primary inductance ($L_P$)\color{blue}, \color{black} and coupling coefficient ($km$) \cite{Transformer_model}. 

\subsection{Design A (no RF choke \& with $L_2C_2$)}
Design A consists of a second harmonic trap ($L_2C_2$) \color{blue},\color{black} which provides short at $2\omega_0$. The $V_{DD}$ is \color{blue} supplied through the balun's center tap \color{black} and $L_{BND}$ is used to model bond-wire inductance ($\approx$ \textit{1 nH}).
Analysis of the schematics is \color{blue} performed in differential-/common-mode scenarios by utilizing the equivalent circuits depicted in Fig. \ref{fig:Design_A}b/c, respectively\color{black} to calculate the unknown parameters: \color{blue} transformer's coupling factor ($km$) and turn ratio ($N$) along with \color{black} $L_P$, $C_2$, $L_2$, and $C_L$.

\begin{figure}[!t]
\captionsetup{font=footnotesize}
\centering
\begin{subfigure}{0.5\textwidth}
\centering
\includegraphics[width=0.5\textwidth]{Images/Design/Design_A_FC.png}
\label{fig:Design_A_FC}
\end{subfigure}
\begin{subfigure}[b]{0.24\textwidth}
\includegraphics[width=1\textwidth]{Images/Design/Design_A_Diff.png}
\label{fig:Design_A_Diff}
\end{subfigure}
\begin{subfigure}[b]{0.24\textwidth}
\includegraphics[width=1\textwidth]{Images/Design/Design_A_Com.png}
\label{fig:Design_A_Com}
\end{subfigure}
\caption{(a) Design A (Balun, $L_2C_2$ and $C_L$), (b) differential-mode, and (c) common-mode equivalent circuit }
\label{fig:Design_A}
\vspace{-0.2in}
\end{figure}

Fig. \ref{fig:Design_A} shows that the drain impedance ($Z_D$) is given by
\vspace{-0.05in}
\begin{equation}
    Z_D=(\frac{1}{\frac{j\omega C_{DS}}{2}}+\frac{1}{Z_{SB}})^{-1}=38.7 \hspace{1mm} \Omega
    \label{eqn:ZD}
\end{equation}
The value of $Z_{SB}$ (impedance of $L_2C_2$ and balun given by (\ref{eqn:Design_A_ZSB})) that will provide $Z_D$ of \textit{38.7} $\Omega$ can be calculated from (\ref{eqn:ZD}) and the value is $\Re(Z_{SB})(\omega_0) =  29.8\hspace{1mm} \Omega$ and $\Im(Z_{SB})(\omega_0) = 16.6\hspace{1mm}\Omega$.
\vspace{-0.05in}
\begin{equation}
\begin{aligned}
    &Z_{SB}=(\frac{1}{Z_B}+\frac{1}{Z_S})^{-1}
    \hspace{1mm}\text{where}, Z_S=2j\omega  L_2+\frac{1}{\frac{j \omega C_2}{2}}, \\
    &Z_B=(\frac{1}{R_P}+\frac{1}{2j \omega  L_m}+j \omega C_P)^{-1}+2j \omega  L_k,\\ &R_P=R_L(\frac{km}{n})^2,C_P=C_L(\frac{n}{km})^2
\label{eqn:Design_A_ZSB}
\end{aligned}
\end{equation}

\color{blue} It is evident from (7) that $C_{DS}$ should resonate with $\Im(Z_{SB})$ to attain high drain impedance at $3\omega_0$. This implies $\Im(Z_{SB})(3\omega_0) = 24.96\hspace{1mm}\Omega$. \color{black} Ideally, $\Re(Z_{SB})(3\omega_0)$ should be \textit{0} to achieve high $3^{rd}$ harmonic impedance\color{blue}. However, \color{black} Fig. \ref{fig:Design_A_Rn_var_1H} depicts that having a larger $\Re(Z_{SB)}(3\omega_0)$ \color{blue} contributes to a \color{black} constant $P_{OUT}$ across the operational bandwidth by \color{blue} flattering the \color{black} real part at the fundamental. Another benefit is that it has \color{blue} a \color{black} more linear reactive part at the fundamental\color{blue}, which is essential \color{black} in CCF operation.
However, this leads to a lower $3^{rd}$ harmonic impedance, as showcased in Fig. \ref{fig:Design_A_Zn_3H}. This also emphasizes  the significance of $C_L$. \color{blue} Thus, \color{black} to achieve \textit{1000} $\Omega$ at $3^{rd}$ harmonic,  $\Re(Z_{SB})(3\omega_0) = 0.5\hspace{1mm}\Omega$, which is obtained from Fig. \ref{fig:Design_A_Zn_3H}. Fig. \ref{fig:Design_A} proves that $L_2$ should have a series resonance with $C_2$ to get short at 2$\omega_0$.

\vspace{-0.05in}
\begin{equation}
    L_2=\frac{1}{4*\omega_0^2*C_2}%=0.73 \hspace{1mm} nH
    \label{eqn:Design_A_2H}
\end{equation}

\begin{figure}[!t]
\captionsetup{font=footnotesize}
\centering
\begin{subfigure}{0.24\textwidth}
\includegraphics[width=1\textwidth]{Images/Design/Design_A_Zn_3H.jpg}
\caption{}
\label{fig:Design_A_Zn_3H}
\end{subfigure}
\begin{subfigure}{0.24\textwidth}
\includegraphics[width=1\textwidth]{Images/Design/Design_A_Rn_var_3H.jpg}
\caption{}
\label{fig:Design_A_Rn_var_3H}
\end{subfigure}
\begin{subfigure}{0.4\textwidth}
\includegraphics[width=1\textwidth]{Images/Design/Design_A_Rn_var_1H.jpg}
\caption{}
\label{fig:Design_A_Rn_var_1H}
\end{subfigure}
\caption{(a) Magnitude of $Z_{D}$ vs $\Re(Z_{SB})$ at $3\omega_0$; (b) Magnitude of $Z_D$ at $3^{rd}$ harmonic for various $R_{SB}(3\omega_0)$; (c) $Z_D$ at fundamental for various $R_{SB}(3\omega_0)$.}
\label{fig:Design_A_Rn_var}
\vspace{-0.2in}
\end{figure}

The five unknowns in the circuit: $km$, $N$, $L_P$, $C_2$, and $C_L$ can be calculated by assuming one of them and using four equations ($\Re(Z_{SB})(\omega_0) =  29.8\hspace{1mm} \Omega$, $\Im(Z_{SB})(\omega_0) = 16.6\hspace{1mm}\Omega$, $\Re(Z_{SB})(3\omega_0) = 0.5\hspace{1mm}\Omega$ and  $\Im(Z_{SB})(3\omega_0) = 24.96\hspace{1mm}\Omega$). In this paper, $km =$ \textit{0.8} is assumed and the remaining unknowns ($N =$ \textit{1.34}, $L_P =$ \textit{2.2 nH}, $C_L =$ \textit{0.9 pF}, $C_2 =$ \textit{1.5 pF}, $L_2 =$ \textit{0.73 nH}) are calculated. The $km$ can be varied to get different sets of results in which $L_P$ is minimal and thereby making it layout friendly.


\subsection{Design B (no RF choke \& no $L_2C_2$)}

\begin{figure}[!t]
\captionsetup{font=footnotesize}
\centering
\begin{subfigure}{0.24\textwidth}
\includegraphics[width=1\textwidth]{Images/Design/Design_B_FC.png}
\caption{}
\label{fig:Design_B_FC}
\end{subfigure}
\begin{subfigure}{0.24\textwidth}
\includegraphics[width=1\textwidth]{Images/Design/Design_C_FC.png}
\caption{}
\label{fig:Design_C_FC}
\end{subfigure}
\begin{subfigure}{0.24\textwidth}
\includegraphics[width=1\textwidth]{Images/Design/Design_D_FC.png}
\caption{}
\label{fig:Design_D_FC}
\end{subfigure}
\caption{(a) Design B (Balun and $C_L$); (b) Design C (Balun, RF Choke, $L_2C_2$ \& $C_L$); (c) Design D (Balun, RF Choke \& $C_L$).}
\label{fig:Design_B_C_D}
\vspace{-0.25in}
\end{figure}

In this design (Fig. \ref{fig:Design_B_FC}), $L_2C_2$ is removed, \color{blue} reducing \color{black} the number of unknowns. Like the previous case, differential mode analysis yields four equations and there are four unknowns, thus leading to a single set of values for $km =$ \textit{0.72}, $N =$ \textit{0.9}, $L_P =$ \textit{0.63 nH}, $C_L =$ \textit{3.96 pF}, unlike design A. $C_{DCP}$ is tuned to provide a short at $2\omega_0$ such that $C_{DCP}$ resonates with $L_P$, $L_{BND}$, and $C_{DS}$, which is obtained from the common-mode analysis. Moreover, $C_{DCP}$ provides RF ground and blocks DC.

\subsection{Design C (with RF choke \& with $L_2C_2$)}
In this design (Fig. \ref{fig:Design_C_FC}), $V_{DD}$ is supplied through RF choke, unlike the previous designs. RF chokes are assumed to have a fixed value of \textit{5 nH}. The differential mode analysis yields four equations similar to design A. Assuming $km =$ \textit{0.8}, the other unknowns are calculated as $N =$ \textit{1.14}, $L_P =$ \textit{2.23 nH}, $C_L =$ \textit{1.10 pF}, $C_2 =$ \textit{1.37 pF}.
Like design A, $C_2$ should resonate out with $L_2$ to get short at $2\omega_0$. Thus, $L_2 =$ \textit{0.8 nH}. 

\subsection{Design D (with RF choke \& no $L_2C_2$)}
 Unlike design C, $L_2C_2$ is removed in this design (Fig. \ref{fig:Design_D_FC}). Also, RF choke should be calculated, assuming $C_{DS}$ resonates with it at $\omega_0$ (RF Choke = \textit{2.35 nH}).
The differential mode analysis yields four equations and the four unknowns ($N =$ \textit{0.84}, $L_P =$ \textit{0.86 nH}, $C_L =$ \textit{3.95 pF}, $km =$ \textit{0.77}) can be calculated.
$C_{DCP}$ can be tuned to provide a short at $2\omega_0$.

\section{Results}
\label{section:Results}

Fig. \ref{fig:Comp_1H_2H_3H}a/b show that all the four designs satisfying the main CCF requirements which \color{blue} are \color{black} decreasing trend of the reactive part at \color{blue} the \color{black} fundamental and increasing trend of the reactive part at $2^{nd}$ harmonic.
From Fig. \ref{fig:Comp_1H}, it is seen that the real part at \color{blue} the \color{black} fundamental is flatter in the bandwidth \textit{2.1 - 2.7 GHz} for design A and C, which, in turn, leads to constant $P_{OUT}$ in the specified bandwidth unlike design B and D. The $L_2C_2$, which acts as a varying capacitor at fundamental hold the reason for this. The designs B and D have a higher reactive part at the fundamental as compared to other designs, which in turn, leads to larger $\gamma$ value (refer Eqn. \ref{eqn_CCF_imp}) and thus, higher \color{blue} peak drain voltage ( Fig. \ref{fig:CCF_wave_VI}). \color{black} Fig. \ref{fig:Comp_1H_2H_3H}b/c show that all the four designs have similar response at $2^{nd}$ and $3^{rd}$ harmonic. 
%The impedance at \textit{2.1 GHz} and \textit{2.7 GHz} is less than \textit{200} $\Omega$ for all the \textit{4} designs.

\begin{figure}[!t]
\captionsetup{font=footnotesize}
\centering
\begin{subfigure}{0.5\textwidth}
\centering
\includegraphics[width=0.65\textwidth]{Images/Output_Network_Comp/Comp_1H.jpg}
\caption{}
\label{fig:Comp_1H}
\end{subfigure}
\begin{subfigure}{0.24\textwidth}
\includegraphics[width=1\textwidth]{Images/Output_Network_Comp/Comp_2H_imag.jpg}
\caption{}
\label{fig:Comp_2H_imag}
\end{subfigure}
\begin{subfigure}{0.24\textwidth}
\includegraphics[width=1\textwidth]{Images/Output_Network_Comp/Comp_3H_Mag.jpg}
\caption{}
\label{fig:Comp_3H_Mag}
\end{subfigure}
\caption{(a) Impedance ($Z_D$) at $1^{st}$ harmonic; (b) Reactive part of $Z_D$ ($\Im(Z_D)$) at $2^{nd}$ harmonic; (c) Magnitude of $Z_D$ ($|Z_D|$) at $3^{rd}$ harmonic.}
\label{fig:Comp_1H_2H_3H}
\vspace{-0.1in}
\end{figure}


The four proposed output networks are tested with an ideal output stage (single transistor \color{blue} acting \color{black} as a current source when turned on) in ADS. Fig. \ref{fig:Comp_Pout_DE} shows there is a large variation in peak $P_{OUT}$ and maximum $\eta_D$ across the operational bandwidth for the design B and D, unlike design A and C. This variation can be attributed to the varying real part at the fundamental in those designs.
So from the simulation, it is seen that design A outperforms other designs since it has the least number of components as well as more constant $P_{\text{OUT}}$ and $\eta_D$ in the specified bandwidth.

\begin{figure}[!t]
\captionsetup{font=footnotesize}
\centering
\begin{subfigure}{0.24\textwidth}
\centering
\includegraphics[width=1\textwidth]{Images/Output_Network_Comp/Comp_DE.jpg}
\caption{}
\label{fig:Comp_DE}
\end{subfigure}
\begin{subfigure}{0.24\textwidth}
\includegraphics[width=1\textwidth]{Images/Output_Network_Comp/Comp_Pout.jpg}
\caption{}
\label{fig:Comp_Pout}
\end{subfigure}
\caption{(a) Maximum drain efficiency across frequency for four different output network; (b) Peak $P_{OUT}$ across frequency for four different output network.}
\label{fig:Comp_Pout_DE}
\vspace{-0.25in}
\end{figure}

\section{Conclusion}
\label{section:Conclusion}
This paper presented CCF's wide operational bandwidth advantage over its class F companion. It detailed the primary requirement of the CCF PA output network, which is if the reactive part of $1^{st}$ harmonic decreases, then the reactive part of $2^{nd}$ harmonic should increase. Furthermore, the procedure to design the four different output networks for the \textit{2.1 -- 2.7 GHz} band was proposed and analyzed in detail.  Consequently, design A, with no RF choke and  including a 2$^{nd}$ harmonic trap, was chosen because it offers relatively constant output RF power in the desired frequency band with the least number of lumped components.

\bibliographystyle{IEEEtran}
\bibliography{disseration.bib}

\end{document}


