\documentclass[conference]{IEEEtran}

\usepackage{cite}


\ifCLASSINFOpdf
  \usepackage[pdftex]{graphicx}

  \graphicspath{{../pdf/}{../jpeg/}}

  \DeclareGraphicsExtensions{.pdf,.jpeg,.png}
\else

\fi

\usepackage{caption}
\usepackage{subcaption}

\usepackage{amsmath}
\usepackage{multirow}
\usepackage{multicol}
\usepackage{mathptmx}
\usepackage{algorithmic}
\usepackage[para]{threeparttable}
\usepackage{array}
\usepackage{color}

\hyphenation{op-tical net-works semi-conduc-tor}


\begin{document}

\title{On-Chip Output Stage Design for a  Continuous Class-F Power Amplifier}

\author{\IEEEauthorblockN{Anil Kumar Kumaran\IEEEauthorrefmark{1},
Masoud Pashaeifar\IEEEauthorrefmark{1},
M. D'Avino\IEEEauthorrefmark{2},
Leo C.N. de Vreede\IEEEauthorrefmark{1}, and 
Morteza S. Alavi\IEEEauthorrefmark{1}} 
\IEEEauthorblockA{\IEEEauthorrefmark{1} Electronic Circuits and Architecture (ELCA) Research Group, Delft University of Technology \\
	\IEEEauthorblockA{\IEEEauthorrefmark{2} Catena B.V., Delft, The Netherlands} Email: a.k.kumaran@tudelft.nl}}

\maketitle

\begin{abstract}
Continuous Class F (CCF) power amplifiers (PAs) overcome Class--F PA's disadvantage of narrow bandwidth by relaxing the short-circuit requirement at the $2^{nd}$ harmonic while still maintaining 90.7\% peak efficiency over the band of interest. This paper proposes four different CCF output networks suitable for on-chip implementation in the 2.1 -- 2.7GHz band and a detailed, step-by-step design procedures are given. The output stage with $2^{nd}$ harmonic trap and no RF choke, is favoured due to its  flat real impedance, low fundamental reactance, and its compact layout. Using a 40nm CMOS process, a passive efficiency of 68\% at 2.4GHz for this structure has been achieved.
\end{abstract}

\vspace{1mm}
% keywords
\begin{IEEEkeywords}
Continuous class F (CCF), Output matching network, Power amplifier (PA), Harmonic termination, Differential-mode analysis, Common-mode analysis. 
\end{IEEEkeywords}

%\vspace{-0.1in}

\ifCLASSOPTIONpeerreview
\begin{center} \bfseries EDICS Category: 3-BBND \end{center}
\fi

\IEEEpeerreviewmaketitle

\section{Introduction}
Today, there is an increased demand for high-speed, low-cost transceivers in cellular/WLAN wireless networks. A CMOS-based solution is considered to be a viable option to address these requirements since it can facilitate the implementation of a System-on-Chip (SoC) solution at a low cost. However, the TX power amplifier (PA) is generally considered as the most challenging block in a CMOS SoC since it must offer high-efficiency operation while meeting the stringent spectral mask requirements of wireless standards. Currently, most of the TXs employ linear class A/B PAs. Nevertheless, their ideal  peak efficiencies are constrained to 50/78.5\%, respectively, due to the relatively large overlap in their drain voltage/current  waveforms (Fig. \ref{fig:wave_VI}a/b).

On the other hand, class--F PAs ideally achieve peak efficiency of 100\% by utilizing harmonic-frequency resonators to short-circuit at even harmonics and open-circuit at odd harmonics leading to non-overlapping drain voltage/current waveforms (Fig. \ref{fig:ICF_wave_VI}). Nonetheless, in reality, controlling all harmonics is challenging and increases the network complexity and component losses, degrading its passive efficiency. Therefore, a practical implementation of class--F PAs employ up to three harmonics and can ideally achieve a peak efficiency up to 90.7\% (Fig. \ref{fig:CF_wave_VI}) \cite{Raab_max_eff}.
\begin{figure}[!t]
\centering
\captionsetup{font=footnotesize}
\begin{subfigure}{0.24\textwidth}
\includegraphics[width=1\textwidth]{Images/Intro/ClassA_shaded.pdf}
\caption{Class--A}
\label{fig:CA_wave_VI}
\end{subfigure}
\begin{subfigure}{0.24\textwidth}
\includegraphics[width=1\textwidth]{Images/Intro/ClassB_shaded.pdf}
\caption{Class--B}
\label{fig:CB_wave_VI}
\end{subfigure}
\begin{subfigure}{0.24\textwidth}
\includegraphics[width=1\textwidth]{Images/Intro/ClassF.pdf}
\caption{Ideal Class--F}
\label{fig:ICF_wave_VI}
\end{subfigure}
\begin{subfigure}{0.24\textwidth}
\includegraphics[width=1\textwidth]{Images/Intro/CF_wave_VI_shaded.pdf}
\caption{Practical Class--F}
\label{fig:CF_wave_VI}
\end{subfigure}
\caption{Class A/B/F drain  voltage/current (V/I) waveforms.}
\label{fig:wave_VI}
\vspace{-0.25in}
\end{figure}
The generalized drain-source-voltage ($V_{DS}$) containing all frequencies up to $3^{rd}$ harmonic \cite{Gen_Vds_eqn} is given by:
\begin{equation}
V_{DS}=\underbrace{1}_{\text{DC}}-\underbrace{\frac{2}{\sqrt{3}} \cos \theta}_{\text{$1^{st}$ harmonic}}+\underbrace{\frac{1}{3 \sqrt{3}} \cos 3 \theta}_{\text{$3^{rd}$ harmonic}}
\label{eqn_CF_V}
\end{equation}
The drain current ($I_{D}$), which is a half-sine wave, is given by
\begin{equation}
I_{D}=\underbrace{\frac{1}{\pi}}_{\text{DC}}+\underbrace{\frac{1}{2} \cos \theta}_{\text{$1^{st}$ harmonic}}+\underbrace{\frac{2}{3 \pi} \cos 2 \theta}_{\text{$2^{nd}$ harmonic}}-\underbrace{\frac{2}{15 \pi} \cos 4 \theta}_{\text{$4^{th}$ harmonic}}
\label{eqn_CCF_I}
\end{equation}
where $\theta$ is the fundamental phase. The load impedances at the  $1^{st}$, $2^{nd}$, and $3^{rd}$ harmonic are represented by $Z_{1f}=\frac{4}{\sqrt{3}}$, $Z_{2f}=0$, and $Z_{3f}=\infty$, respectively. As depicted in Fig. \ref{fig:CF_wave_VI},  one of the key advantages of class--F PAs is that its normalised peak drain voltage is two. Nonetheless, these traditional class--F suffer operational bandwidth limitations (typically up to 10\%) due to their dependence on ideal short and open circuit harmonic terminations. To realize larger bandwidths, the continuous class--F (CCF) PA has been proposed \cite{CCF_reason}. In this paper, we present four different output network topologies, all suited to CCF operation. 

%This paper is organized as follows. Section \ref{section:CCF} briefly explains the equations governing the operation of CCF. Section \ref{section:ON} discusses the design procedure of different CCF output networks. Section \ref{section:Results} compares the proposed four output stages, and presents layout of the chosen design. Section \ref{section:Conclusion} concludes this paper.

\section{Continuous Class-F}
\label{section:CCF}
\vspace{-0.05in}
Compared to class--F, the CCF operation allows an extra imaginary part at both its $1^{st}$/$2^{nd}$ harmonic of the voltage waveform. Thus, the generalized $V_{DS}$ for CCF is given by \cite{ECCF_Carrubba}:
\begin{equation}
V_{DS}=\underbrace{1}_{\text{DC}}-\underbrace{\frac{2}{\sqrt{3}} \cos \theta-\gamma \sin \theta}_{\text{$1^{st}$ harmonic}}+\underbrace{\frac{7 \gamma}{6 \sqrt{3}} \sin 2 \theta}_{\text{$2^{nd}$ harmonic}}+\underbrace{\frac{1}{3 \sqrt{3}} \cos 3 \theta}_{\text{$3^{rd}$ harmonic}}
\label{eqn_CCF_V}
\end{equation}
where $\gamma$ is an empirical parameter. As in class--F, $I_{D}$ is a half sinusoid [given in (\ref{eqn_CCF_I})]. These equations indicate that $V_{DS}$/$I_{D}$ are chosen so that no power is dissipated at higher harmonics. Also, mathematically, $\gamma$ in (\ref{eqn_CCF_V}) does not affect the drain efficiency ($\eta_D$). But it does affect the peak drain voltage, as depicted in Fig. \ref{fig:CCF_wave_VI}. However, $I_{D}$ depends on $V_{DS}$, degrading $\eta_D$ as $\gamma$ increases.


Fig. \ref{fig:CCF_wave_VI} implies that at $\gamma$ = 0, the waveform is like class F. For the $\gamma$ values between -1 and 1, $V_{DS}$ remains positive, enabling CCF PA to meet linearity requirements. However, CCF PAs suffer from large peak drain voltage which can be as high as 3.34 times the supply ($V_{DD}$) when $\gamma$ = -1 or 1. Using (\ref{eqn_CCF_V}) and (\ref{eqn_CCF_I}), the load impedances of CCF  are $Z_{1f}=\frac{4}{\sqrt{3}}+j 2 \gamma$, $Z_{2f}=0-j \frac{\pi}{2} \frac{7 \sqrt{3}}{6} \gamma$, and $Z_{3f}=\infty$ \cite{CCFDesign_ali}.


In CCF, $Z_{3f}$ remains open-circuited similar to class--F. Meanwhile, $Z_{1f}$ and $Z_{2f}$ have a reactive part, unlike class--F. From CCF load impedances and Fig. \ref{fig:CCF_SC}, it is perceived that if the reactive part of $Z_{1f}$ changes from inductive to capacitive, then the reactive part of $Z_{2f}$  needs to change from capacitive to inductive or vice-versa across the bandwidth, to achieve CCF operation. In the next section, the design procedure for the proposed four output networks is meticulously presented. 


\begin{figure}[!t]
\captionsetup{font=footnotesize}
\centering
\begin{subfigure}{0.24\textwidth}
\centering
\includegraphics[width=1\textwidth]{Images/CCF/CCF_wave_VI.pdf}
\caption{}
\label{fig:CCF_wave_VI}
\end{subfigure}
\begin{subfigure}{0.24\textwidth}
\includegraphics[width=1\textwidth]{Images/CCF/CCF_SC.jpg}
\caption{}
\label{fig:CCF_SC}
\end{subfigure}
\caption{(a) $V_{DS}$ and $I_D$ waveform for CCF with -1.5 $<$ $\gamma$ $<$ 1.5, and (b) Variation of $Z_{1f}$, $Z_{2f}$, $Z_{3f}$ for -1 $<$ $\gamma$ $<$ 1.}
\label{fig:CCF_VI_SC}
\vspace{-0.2in}
\end{figure}

 

\section{Design of Output Network for CCF}
\label{section:ON}
In this paper, the push-pull (differential) structure is chosen for the PA, mainly because it decouples the odd harmonics ($1^{st}$/$3^{rd}$) from even harmonic ($2^{nd}$) impedance. Moreover, it suppresses supply/substrate noise, second-order nonlinearities and doubles the RF output power for a given breakdown voltage. 
In this work, the targeted peak RF power is 27dBm  while its operational bandwidth is 2.1--2.7GHz with $V_{DD}$ = 2.7V. To achieve this, a  fundamental differential impedance of 38.7$\Omega$ should be presented to the drains of the transistors across the 2.1--2.7GHz band, which can be obtained as follows:
\vspace{-0.05in}
\begin{equation}
\begin{aligned}
&V_{FUND-diff}=2\times V_{FUND-single}=2\times\frac{2}{\sqrt{3}} V_{DD}=6.24 \hspace{1mm}V\\
&R_{D-diff}=\frac{V_{FUND-diff}^{2}}{2\times\hspace{1mm} P_{OUT_{max}}}=38.7 \hspace{1mm} \Omega
\label{eqn_diff_imp}
\end{aligned}
\end{equation}
The requirements of the output network for CCF is exhibited in Tab. \ref{tab:Output_Network_Requirements}. The PA operates in class--F mode at the center frequency $\omega_0$ (2.4GHz) with a short at $2\omega_0$ (4.8GHz) and an open at $3\omega_0$ (7.2GHz). But, for all other frequencies, the PA performs in the CCF mode. 

\setlength{\arrayrulewidth}{0.1mm}
\setlength{\tabcolsep}{1pt}
\renewcommand{\arraystretch}{1.2}
\begin{table}[!t]
\centering
\captionsetup{font=footnotesize}
\caption{OUTPUT NETWORK SPECIFICATIONS.}
\label{tab:Output_Network_Requirements}
\resizebox{\linewidth}{!}{%
\begin{tabular}{|c|c|c|c|}
\hline
\begin{tabular}[c]{@{}c@{}}\textbf{Class of} \\\textbf{Operation}\end{tabular} & \begin{tabular}[c]{@{}c@{}}\textbf{First Harmonic} \\($\omega$)\end{tabular} & \begin{tabular}[c]{@{}c@{}}\textbf{Second Harmonic} \\(2$\omega$)\end{tabular} & \begin{tabular}[c]{@{}c@{}}\textbf{Third Harmonic} \\(3$\omega$)\end{tabular} \\ \hline
\multirow{2}{*}{\textbf{\begin{tabular}[c]{@{}c@{}}Class F \\ (2.4GHz)\end{tabular}}} & $\Re(Z_D)$ = 38.7$\Omega$ & $\Re(Z_D)$= 0$\Omega$ & \multirow{2}{*}{\begin{tabular}[c]{@{}c@{}}$|Z_D|\approx$ 1000$\Omega$\end{tabular}} \\ \cline{2-3}
 & $\Im(Z_D)$ = 0$\Omega$ & $\Im(Z_D)$ = 0$\Omega$ &  \\ \hline
\multirow{2}{*}{\textbf{\begin{tabular}[c]{@{}c@{}}\\CCF \\ (2.1--2.7GHz)\end{tabular}}} & $\Re(Z_D)$ = 38.7$\Omega$ & $\Re(Z_D)$ = 0$\Omega$ & \multirow{2}{*}{\begin{tabular}[c]{@{}c@{}}\\$|Z_D|\approx$ 1000$\Omega$ \end{tabular}} \\ \cline{2-3}
 & \begin{tabular}[c]{@{}c@{}}$\Im(Z_D)$ $\rightarrow$ + to -\\ OR\\ $\Im(Z_D)$ $\rightarrow$ - to +\end{tabular} & \begin{tabular}[c]{@{}c@{}} $\Im(Z_D)$ $\rightarrow$ - to +\\ OR\\ $\Im(Z_D)$ $\rightarrow$ + to -\end{tabular} &  \\ \hline
\end{tabular}%
 }
\vspace{-0.25in}
\end{table}

The four different  output networks are proposed based on, lumped passive components, see Fig. \ref{fig:Design_A_B_C_D}a/b/c/d. All the designs include a balun and load capacitance ($C_L$). The balun converts the balanced (differential) signal to its single-ended companion, whereas $C_L$ adjusts the $3^{rd}$ harmonic impedance. The power transistor's drain-source capacitance (assumed $C_{DS}=1.87$pF) is absorbed into the output network to reduce its impact on the PA performance. The balun is modeled using an ideal transformer, consisting of magnetizing inductance ($L_m$), leakage inductance ($L_k$), primary inductance ($L_P$), and coupling coefficient ($k_m$) \cite{Transformer_model}. 

\subsection{Design A (no RF Choke \& with $L_2C_2$)}
Design A (Fig. \ref{fig:Design_A_B_C_D}a) includes a $2^{nd}$ harmonic trap ($L_2C_2$), which provides short at $2\omega_0$. The $V_{DD}$ is supplied through the balun's center tap, and $L_{BND}$ is used to model the bond-wire inductance ($\approx$1nH). Analysis of the schematics is performed in differential-/common-mode scenarios by utilizing the equivalent circuits depicted in Fig. \ref{fig:Design_A_B_C_D}a.1/a.2, respectively, to calculate the unknown parameters: transformer's coupling factor ($k_m$) and turn ratio ($n$) along with  $L_P$, $C_2$, $L_2$, and $C_L$. Fig. \ref{fig:Design_A_B_C_D}a.1 shows that the drain impedance ($Z_D$) is given by
\vspace{-0.05in}
\begin{equation}
	Z_D=\left(\frac{1}{\frac{j\omega C_{DS}}{2}}+\frac{1}{Z_{SB}}\right)^{-1}=38.7 \hspace{1mm} \Omega
	\label{eqn:ZD}
\end{equation}

\begin{figure}[!t]
\captionsetup{font=footnotesize}
\centering
\begin{subfigure}{0.5\textwidth}
\centering
\includegraphics[width=1\textwidth]{Images/Design/Design_A_FC.pdf}
\caption{}
\label{fig:Design_A_FC}
\end{subfigure}
\begin{subfigure}[b]{0.35\textwidth}
\includegraphics[width=1\textwidth]{Images/Design/Design_A_Diff.pdf}
\caption*{(a.1)}
\label{fig:Design_A_Diff}
\end{subfigure}
\begin{subfigure}[b]{0.35\textwidth}
\includegraphics[width=1\textwidth]{Images/Design/Design_A_Com.pdf}
\caption*{(a.2)}
\label{fig:Design_A_Com}
\end{subfigure}

\begin{subfigure}{0.2\textwidth}
\centering
\includegraphics[width=1\textwidth]{Images/Design/Design_B_FC.pdf}
\caption{}
\label{fig:Design_B_FC}
\end{subfigure}
\begin{subfigure}{0.26\textwidth}
\centering
\includegraphics[width=1\textwidth]{Images/Design/Design_C_FC.pdf}
\caption{}
\label{fig:Design_C_FC}
\end{subfigure}
\centering
\begin{subfigure}{0.24\textwidth}
\centering
\includegraphics[width=1\textwidth]{Images/Design/Design_D_FC.pdf}
\caption{}
\label{fig:Design_D_FC}
\end{subfigure}
\caption{(a) Design A (Balun, $L_2C_2$ and $C_L$), (a.1) odd-mode, (a.2) even-mode equivalent circuit (b) Design B (Balun and $C_L$), (c) Design C (Balun, RF choke, $L_2C_2$ \& $C_L$), and (d) Design D (Balun, RF choke \& $C_L$).}
\label{fig:Design_A_B_C_D}
\vspace{-0.2in}
\end{figure}

The value of $Z_{SB}$ [impedance of $L_2C_2$ and balun given by (\ref{eqn:Design_A_ZSB})] that will provide $Z_D$ of 38.7$\Omega$ can be calculated from (\ref{eqn:ZD}) and the value is $\Re(Z_{SB})(\omega_0) =  29.8\Omega$ and $\Im(Z_{SB})(\omega_0) = 16.6\Omega$.
\vspace{-0.05in}
\begin{equation}
\begin{aligned}
    &Z_{SB}=\left(\frac{1}{Z_B}+\frac{1}{Z_S}\right)^{-1}
    \hspace{1mm}\text{where}, Z_S=2j\omega  L_2+\frac{1}{\frac{j \omega C_2}{2}}, \\
    &Z_B=\left(\frac{1}{R_P}+\frac{1}{2j \omega  L_m}+j \omega C_P\right)^{-1}+2j \omega  L_k,\\ &R_P=R_L\left(\frac{k_m}{n}\right)^2,C_P=C_L\left(\frac{n}{k_m}\right)^2
\label{eqn:Design_A_ZSB}
\end{aligned}
\end{equation}

It is evident from (\ref{eqn:ZD}) that $C_{DS}$ should resonate with $\Im(Z_{SB})$ to attain high drain impedance at $3\omega_0$. This implies $\Im(Z_{SB})(3\omega_0) = 24.96\Omega$. Ideally, $\Re(Z_{SB})(3\omega_0)$ should be $0\Omega$ to achieve high $3^{rd}$ harmonic impedance. However, Fig. \ref{fig:Design_A_Rn_var_1H} depicts that having a larger $\Re(Z_{SB})(3\omega_0)$ contributes to a constant $P_{OUT}$ across the operational bandwidth by flattening the real part at the fundamental. Another benefit is that it has a  more linear reactive part at the fundamental, which is essential in CCF operation.
However, this leads to a lower $3^{rd}$ harmonic impedance, as showcased in Fig. \ref{fig:Design_A_Rn_var_3H}. This also emphasizes  the significance of $C_L$. Thus, to achieve an open circuit at the $3^{rd}$ harmonic, an impedance of approximately 1000$\Omega$ is required which is obtained with  $\Re(Z_{SB})(3\omega_0) = 0.5\Omega$ (Fig. \ref{fig:Design_A_Zn_3H}). Fig. \ref{fig:Design_A_B_C_D}a.2 proves that $L_2$ should have a series resonance with $C_2$ to get short at 2$\omega_0$.

\vspace{-0.05in}
\begin{equation}
    L_2=\frac{1}{4\times\omega_0^2\times C_2}%=0.73 \hspace{1mm} nH
    \label{eqn:Design_A_2H}
\end{equation}

\begin{figure}[!t]
\captionsetup{font=footnotesize}
\centering
\begin{subfigure}{0.24\textwidth}
\includegraphics[width=1\textwidth]{Images/Design/Design_A_Zn_3H.pdf}
\caption{}
\label{fig:Design_A_Zn_3H}
\end{subfigure}
\begin{subfigure}{0.24\textwidth}
\includegraphics[width=1\textwidth]{Images/Design/Design_A_Rn_var_3H.pdf}
\caption{}
\label{fig:Design_A_Rn_var_3H}
\end{subfigure}
\begin{subfigure}{0.5\textwidth}
\centering
\includegraphics[width=0.55\textwidth]{Images/Design/Design_A_Rn_var_1H.pdf}
\caption{}
\label{fig:Design_A_Rn_var_1H}
\end{subfigure}
\caption{(a) Magnitude of $Z_{D}$ vs real part of $Z_{SB}$ ($\Re(Z_{SB})$) at $3\omega_0$, (d) Magnitude of $Z_D$ at $3^{rd}$ harmonic for various $R_{SB}(3\omega_0)$, and (c) $Z_D$ at fundamental for various $R_{SB}(3\omega_0)$.}
\label{fig:Design_A_Rn_var}
\vspace{-0.2in}
\end{figure}

The five unknowns in the circuit: $k_m$, $n$, $L_P$, $C_2$, and $C_L$ can be calculated by assuming one of them and using four equations [$\Re(Z_{SB})(\omega_0) =  29.8 \Omega$, $\Im(Z_{SB})(\omega_0) = 16.6\Omega$, $\Re(Z_{SB})(3\omega_0 = 0.5\Omega$ and  $\Im(Z_{SB})(3\omega_0) = 24.96\Omega$]. In this paper, $k_m=$ 0.69 is assumed and the remaining unknowns ($n =$ 1, $L_P =$ 0.8nH, $C_L =$ 2.4pF, $C_2 =$ 0.8pF, $L_2 =$ 1.4nH) are calculated. $k_m$ can be varied to get different sets of results in which $L_P$ is minimal and thereby making the design layout friendly.


\subsection{Design B (no RF Choke \& no $L_2C_2$)}

%\begin{figure}[!t]
%\captionsetup{font=footnotesize}
%\centering
%\begin{subfigure}{0.2\textwidth}
%\centering
%\includegraphics[width=1\textwidth]{Images/Design/Design_B_FC.pdf}
%\caption{}
%\label{fig:Design_B_FC}
%\end{subfigure}
%\begin{subfigure}{0.26\textwidth}
%\centering
%\includegraphics[width=1\textwidth]{Images/Design/Design_C_FC.pdf}
%\caption{}
%\label{fig:Design_C_FC}
%\end{subfigure}
%\centering
%\begin{subfigure}{0.24\textwidth}
%\centering
%\includegraphics[width=1\textwidth]{Images/Design/Design_D_FC.pdf}
%\caption{}
%\label{fig:Design_D_FC}
%\end{subfigure}
%\caption{(a) Design B (Balun and $C_L$), (b) Design C (Balun, RF choke, $L_2C_2$ \& $C_L$), and (c) Design D (Balun, RF choke \& $C_L$).}
%\label{fig:Design_B_C_D}
%\vspace{-0.25in}
%\end{figure}

In this design (Fig. \ref{fig:Design_B_FC}), $L_2C_2$ is removed, reducing the number of unknowns. Like the previous case, differential mode analysis yields four equations, and there are four unknowns, thus leading to a single set of values for $k_m=$ 0.72, $n =$ 0.9, $L_P =$ 0.63nH, $C_L =$ 3.96pF, unlike design A. $C_{DDP}$ is tuned to provide a short at $2\omega_0$ such that $C_{DDP}$ resonates with $L_P$, $L_{BND}$, and $C_{DS}$, which is obtained from the common-mode analysis. Moreover, $C_{DDP}$ provides RF ground and blocks DC.

\subsection{Design C (with RF Choke \& with $L_2C_2$)}
In this design (Fig. \ref{fig:Design_C_FC}), $V_{DD}$ is supplied through a RF choke, unlike the previous designs. RF chokes are assumed to have a fixed value of 5nH. The differential mode analysis yields four equations similar to design A. Assuming $k_m=$ 0.73, the other unknowns are calculated as $n =$ 0.93, $L_P =$ 0.95nH, $C_L =$ 2.5pF, $C_2 =$ 0.61pF.
Like design A, $C_2$ should resonate out with $L_2$ to get short at $2\omega_0$. Thus, $L_2 =$ 1.8nH. 

\subsection{Design D (with RF Choke \& no $L_2C_2$)}
 Unlike design C, $L_2C_2$ is removed in this design (Fig. \ref{fig:Design_D_FC}). Also, the RF choke should be calculated, assuming $C_{DS}$ resonates with it at $\omega_0$ (RF choke = 2.35nH).
The differential mode analysis yields four equations and the four unknowns ($n =$ 0.84, $L_P =$ 0.86nH, $C_L =$ 3.95pF, $k_m=$ 0.77) can be calculated.
$C_{DDP}$ can be tuned to provide a short at $2\omega_0$.

\begin{figure}[!t]
	\captionsetup{font=footnotesize}
	\centering
	\begin{subfigure}{0.5\textwidth}
		\centering
		\includegraphics[width=0.55\textwidth]{Images/Output_Network_Comp/Comp_1H.pdf}
		\caption{}
		\label{fig:Comp_1H}
	\end{subfigure}
	\begin{subfigure}{0.24\textwidth}
		\includegraphics[width=1\textwidth]{Images/Output_Network_Comp/Comp_2H_imag.pdf}
		\caption{}
		\label{fig:Comp_2H_imag}
	\end{subfigure}
	\begin{subfigure}{0.24\textwidth}
		\includegraphics[width=1\textwidth]{Images/Output_Network_Comp/Comp_3H_Mag.pdf}
		\caption{}
		\label{fig:Comp_3H_Mag}
	\end{subfigure}
	\caption{(a) Impedance ($Z_D$) at $1^{st}$ harmonic, (b) Reactive part of $Z_D$ ($\Im(Z_D)$) at $2^{nd}$ harmonic, and (c) Magnitude of $Z_D$ ($|Z_D|$) at $3^{rd}$ harmonic.}
	\label{fig:Comp_1H_2H_3H}
	\vspace{-0.1in}
\end{figure}

\section{Results}
\label{section:Results}

Fig. \ref{fig:Comp_1H_2H_3H} shows that all the four designs satisfy the main CCF requirement, which is a decreasing trend of the reactive part at the fundamental and increasing trend of the reactive part at $2^{nd}$ harmonic. As illustrated in Fig. \ref{fig:Comp_1H}, the real part in the desired frequency band is flatter for design A/C, which leads to constant $P_{OUT}$, unlike design B/D. Moreover, to fully accomplish CCF operation, the series $L_2C_2$, which resonates at the $2^{nd}$ harmonic, acts as a capacitor at the desired band. The designs B/D have a higher reactive part at the fundamental than other designs, contributing to a larger $\gamma$ value and, thus, higher peak drain voltage (Fig. \ref{fig:CCF_wave_VI}). Fig. \ref{fig:Comp_1H_2H_3H}b/c show that all designs have a similar response at $2^{nd}$/$3^{rd}$ harmonics. Nevertheless, design A generates a more constant $P_{\text{OUT}}$ in the band of interest and has minimal passive components. 

To verify the proposed output passive stage design approach,  design A is laid out in an LP 40nm CMOS process (Fig. \ref{fig:ON_X1}) with a 600$\mu m$ balun. Subsequently, design A is simulated in three different ways: as a lossless network, with losses using a quality factor of 7/13/100 for the inductors/balun/capacitors, respectively, and as the proposed layout. The ADS (Momentum) simulation results are depicted in Fig. \ref{fig:Comp_Pout_DE}, which indicates, there is a relatively good agreement between the lossy and the proposed layout, reaching 68\% passive efficiency at 2.4GHz. Moreover, using an ideal push-pull PA, the proposed design achieves peak RF output power/drain efficiency of more than 25dBm/62\%, respectively, over the desired frequency band. The proposed design is compared with other State-of-the-Art CCF CMOS PAs in Tab. \ref{tab:State_of_Art}. 

\begin{figure}[!t]
\centering
\captionsetup{font=footnotesize}
\includegraphics[width=1\linewidth]{Images/Output_Network_Comp/Balun_V2.jpg}
\caption{Layout of the design A (balun, $L_2C_2$ and $C_L$).}
\label{fig:ON_X1}
\vspace{-0.1in}
\end{figure}

\begin{figure}[!t]
\captionsetup{font=footnotesize}
\centering
\begin{subfigure}{0.24\textwidth}
\centering
\includegraphics[width=1\textwidth]{Images/Output_Network_Comp/Comp_PasEff_loss_layout_km0p69.pdf}
\caption{}
\label{fig:Comp_PasEff_loss_layout_km0p69}
\end{subfigure}
\begin{subfigure}{0.24\textwidth}
\includegraphics[width=1\textwidth]{Images/Output_Network_Comp/Comp_Pout_DE_loss_layout_km0p69.pdf}
\caption{}
\label{fig:Comp_Pout_DE_loss_layout_km0p69}
\end{subfigure}
\caption{(a) Passive efficiency, and (b) Maximum $P_{OUT}$ and drain efficiency versus frequency band.}
\label{fig:Comp_Pout_DE}
\end{figure}

\setlength{\arrayrulewidth}{0.1mm}
\setlength{\tabcolsep}{2pt}
\renewcommand{\arraystretch}{1.2}
\begin{table}[!t]
\centering
\captionsetup{font=footnotesize}
\caption{PERFORMANCE COMPARISON WITH CCF CMOS PAS}
\label{tab:State_of_Art}
\begin{threeparttable}
\resizebox{\linewidth}{!}{%
\begin{tabular}{|c|c|c|c|c|c|c|}
\hline
Ref. & \textbf{\begin{tabular}[c]{@{}c@{}}This\\ work\tnote{*}\end{tabular}} & \begin{tabular}[c]{@{}c@{}}RFIT'20\\ \cite{RFIT20}\end{tabular} & \begin{tabular}[c]{@{}c@{}}APMC'19\\ \cite{APMC19}\end{tabular} & \begin{tabular}[c]{@{}c@{}}RFIC'18\\ \cite{RFIC18}\end{tabular} & \begin{tabular}[c]{@{}c@{}}ISSCC'18\\ \cite{ISSC18}\end{tabular} & \begin{tabular}[c]{@{}c@{}}RFIC'17\\ \cite{RFIC17}\end{tabular} \\ \hline
Tech. & \textbf{40nm} & 65nm & 90nm & 45nm SOI & 65nm & 65nm \\ \hline
\begin{tabular}[c]{@{}c@{}}Freq.\\ (GHz)\end{tabular} & \textbf{2.1--2.7} & 24--30 & 26--31 & 28--39 & 26.5--31\tnote{+}  & 26--34 \\ \hline
\begin{tabular}[c]{@{}c@{}}Max. Pout\\ (dBm)\end{tabular} & \textbf{25.7} & 18.4 & 15.5 & 18.5 & 15.6 & 14.8 \\ \hline
\begin{tabular}[c]{@{}c@{}}Max. PAE\\ (\%)\end{tabular} & \textbf{52.3} & 37.7 & 32 & 45.7 & 41 & 46.4 \\ \hline
%Topology & \textbf{CCF} & CCF & CCF & \begin{tabular}[c]{@{}c@{}}Hybrid of\\ CCF/F$^{-1}$\end{tabular} & CCF & CCF \\ \hline
\end{tabular}%
 }
\begin{tablenotes}
  \item[*] Momentum simulation with ideal device.
  \item[+] Estimate from figure.
  \end{tablenotes}
\end{threeparttable} 
\vspace{-0.25in}
\end{table}

%\setlength{\arrayrulewidth}{0.1mm}
%\setlength{\tabcolsep}{2pt}
%\renewcommand{\arraystretch}{1.2}
%\begin{table}[!t]
%\centering
%\captionsetup{font=footnotesize}
%\caption{PERFORMANCE COMPARISON WITH CCF CMOS PAS}
%\label{tab:Output_Network_Requirements}
%\resizebox{\linewidth}{!}{%
%\begin{tabular}{|c|c|c|c|c|c|}
%\hline
%Ref. & Tech. & \begin{tabular}[c]{@{}c@{}}Freq. \\ (GHz)\end{tabular} & \begin{tabular}[c]{@{}c@{}}Max. Pout \\ (dBm)\end{tabular} & \begin{tabular}[c]{@{}c@{}}Max. PAE \\ (\%)\end{tabular} & Topology \\ \hline
%\textbf{\begin{tabular}[c]{@{}c@{}}This\\ work\end{tabular}} & \textbf{40nm} & \textbf{2.1--2.7} & \textbf{25.7} & \textbf{52.3} & \textbf{CCF} \\ \hline
%\begin{tabular}[c]{@{}c@{}}RFIT'20\\ \cite{RFIT20}\end{tabular} & 65nm & 24--30 & 18.4 & 37.7 & CCF \\ \hline
%\begin{tabular}[c]{@{}c@{}}APMC'19\\ \cite{APMC19}\end{tabular} & 90nm & 26--31 & 15.5 & 32 & CCF \\ \hline
%\begin{tabular}[c]{@{}c@{}}RFIC'18\\ \cite{RFIC18}\end{tabular} & \begin{tabular}[c]{@{}c@{}}45nm \\ SOI\end{tabular} & 28--39 & 18.5 & 45.7 & \begin{tabular}[c]{@{}c@{}}Hybrid of\\ CCFF-1\end{tabular} \\ \hline
%\begin{tabular}[c]{@{}c@{}}ISSCC'18\\ \cite{ISSC18}\end{tabular} & 65nm & 26.5--31* & 15.6 & 41 & CCF \\ \hline
%\begin{tabular}[c]{@{}c@{}}RFIC'17\\ \cite{RFIC17}\end{tabular} & 65nm & 26--34 & 14.8 & 46.4 & CCF \\ \hline
%\end{tabular}%
%}
%\vspace{-0.25in}
%\end{table}


\section{Conclusion}
\label{section:Conclusion}
In this paper, the primary advantage of CCF PAs over its class--F companion is presented. The paper details the critical requirement of the CCF PA output network, that is if the $1^{st}$ harmonic reactive part decreases, then its $2^{nd}$ harmonic reactive part should increase to maintain CCF's peak efficiency in the band of interest. Furthermore, the design procedure of various output networks for the 2.1 -- 2.7GHz band is presented, and the proposed passive output stages are synthesized.  Consequently, design A, with no RF choke and a 2$^{nd}$ harmonic trap, is chosen due to its relatively constant output RF power in the desired frequency band with minimal on-chip passive components. This design is laid out in a 40nm CMOS while achieving a 68\% passive efficiency at 2.4GHz.

\bibliographystyle{IEEEtran}
\bibliography{ISCASvABS.bib}

\end{document}


